

% Report Template
% John Godlee (johngodlee@gmail.com)
% Compile with XeLaTeX

\documentclass[a4paper, 11pt]{article}
%
\usepackage{amsmath}   % Better maths and more symbols
%
\usepackage{geometry}  % Set margins
\geometry{left=1.2cm,
	  right=1.2cm,
	  top=2.2cm,
  	  bottom=2cm}
\parskip 0.5cm
%
\usepackage{pdflscape} % Allow landscape pages nested in pdf
%
\usepackage{graphics}  % Insert images easily
\usepackage{graphicx}
\graphicspath{ {img/} }
\usepackage{float}  % Fancy graphics placement [H] [H!] arguments
%
\usepackage{caption}
%
\usepackage{multirow}  % Table cells spanning multiple rows
%
\usepackage{enumerate}
%
\usepackage{natbib}    % Bibliography management - Use author/date citations
\bibliographystyle{agsmnourl}  % Use my custom agsm bibliography template which never includes URLs in articles
\usepackage{url}
\usepackage{cite}
%
\usepackage{lineno}
\linenumbers
%
\usepackage{eurosym}
%
\usepackage{color} 
\newcommand{\todo}[1]{\textcolor{red}{#1}}   % \todo{NOTE TO SELF WRITTEN IN RED}
%
\title{Geographically and genetically distinct populations of scots pine (\textit{Pinus sylvestris}) do not differ in resistance to damage the large pine weevil (\textit{Hylobius abietis}), a common garden translocation study}
\author{John L. Godlee}


%-----------------------------------------------
\begin{document}
%-----------------------------------------------

\maketitle{}

\begin{abstract}
\end{abstract}

\section*{Introduction}

The large pine weevil (\textit{Hylobius abietis} L. Coleoptera: Curculionidae) is a common pest of newly planted conifer plantations in Europe, generally causing damage to saplings up to five years old \citep{Dillon2008}. Adult weevils emerge from tree stumps and feed on the bark and buds of coniferious saplings. Circular lesions on the bark and buds may cause a reduction in growth rate, stem deformation and an increased susceptibility to infection by airborne diseases of trees \citep{Leather1999}. A single adult weevil can damage several plants over the course of a season, with \textasciitilde{}50\% sapling mortality observed across plantation sites in the UK and Ireland \citep{Heritage2001}. On commercial conifer plantations, \textit{H. abietis} causes annual economic losses of \euro{}140 million \textit{per annum} in Europe, of which \euro{}2.75 million occurs in the UK \citep{Evans2015}. \textit{H. abietis} \todo{benefit from the env. created by the clearfelling of forest}. The potential for climate change to enhance the damage caused by \textit{H. abietis}, by reducing life cycle length and encouraging migration into previously weevil free areas, has prompted discussion of the effectiveness of current management practices and possible alternatives \citep{}.

Control management of \textit{H. abietis} currently relies on a variety of chemical, biological and physical measures, with integrated pest management schemes yielding greater success \citep{}. \todo{MORE ON METHODS OF CONTROL}

\todo{Scots pine is a pretty cool tree}

Rather than the application of costly pesticides or biological control agents, breeding weevil resistant \textit{P. sylvestris} varieties for commercial plantation forestry may 


TEXT

%-----------------------------------------------
\end{document}
%-----------------------------------------------


