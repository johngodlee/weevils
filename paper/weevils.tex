

% Report Template
% John Godlee (johngodlee@gmail.com)
% Compile with XeLaTeX

\documentclass[a4paper, 11pt]{article}
%
\usepackage{amsmath}   % Better maths and more symbols
%
\usepackage{geometry}  % Set margins
\geometry{left=1.2cm,
	  right=1.2cm,
	  top=2.2cm,
  	  bottom=2cm}
\parskip 0.5cm
%
\usepackage{pdflscape} % Allow landscape pages nested in pdf
%
\usepackage{graphics}  % Insert images easily
\usepackage{graphicx}
\graphicspath{ {img/} }
\usepackage{float}  % Fancy graphics placement [H] [H!] arguments
%
\usepackage{caption}
%
\usepackage{multirow}  % Table cells spanning multiple rows
%
\usepackage{enumerate}
%
\usepackage{natbib}    % Bibliography management - Use author/date citations
\bibliographystyle{agsmnourl}  % Use my custom agsm bibliography template which never includes URLs in articles
\usepackage{url}
\usepackage{cite}
%
\usepackage{lineno}
\linenumbers
%
\usepackage{eurosym}
%
\usepackage{color} 
\newcommand{\todo}[1]{\textcolor{red}{#1}}   % \todo{NOTE TO SELF WRITTEN IN RED}
%
\title{Geographically and genetically distinct populations of scots pine (\textit{Pinus sylvestris}) do not differ in resistance to damage the large pine weevil (\textit{Hylobius abietis}), a common garden translocation study}
\author{John L. Godlee}


%-----------------------------------------------
\begin{document}
%-----------------------------------------------

\maketitle{}

\begin{abstract}
\end{abstract}

\section*{Introduction}

The large pine weevil (\textit{Hylobius abietis} L. Coleoptera: Curculionidae) is a common pest of newly planted conifer plantations in Europe, generally causing damage to saplings up to five years old \citep{Dillon2008}. Adult weevils emerge from tree stumps and feed on the bark and buds of coniferious saplings. Circular lesions on the bark and buds may cause a reduction in growth rate, stem deformation and an increased susceptibility to infection by airborne diseases of trees \citep{Leather1999}. A single adult weevil can damage several plants over the course of a season, with \textasciitilde{}50\% sapling mortality observed across plantation sites in the UK and Ireland \citep{Heritage2001}. On commercial conifer plantations, \textit{H. abietis} causes annual economic losses of \euro{}140 million \textit{per annum} in Europe, of which \euro{}2.75 million occurs in the UK \citep{Evans2015}. \textit{H. abietis} \todo{benefit from the env. created by the clearfelling of forest}. The potential for climate change to enhance the damage caused by \textit{H. abietis}, by reducing life cycle length and encouraging migration into previously weevil free areas, has prompted discussion of the effectiveness of current management practices and possible alternatives \citep{}.

Control management of \textit{H. abietis} currently relies on a variety of chemical, biological and physical measures, with integrated pest management schemes yielding greater success \citep{}. \todo{MORE ON METHODS OF CONTROL}

\todo{Scots pine is a pretty cool tree}

Rather than the application of costly pesticides or biological control agents, breeding weevil resistant \textit{P. sylvestris} varieties, or identifying existing genetically distinct population of \textit{P. sylvestris} for commercial plantation forestry may provide a low cost method of reducing economic losses from weevil damage.

Here, we conducted a common garden experiment to assess the resistance of \textit{P. sylvestris} saplings grown from seed collected in populations across the natural longitudinal range of Caledonian Pine to damage from the large pine weevil \textit{H. abietis}. 


\section*{Materials \& Methods}

\subsection*{Study species}

Scots pine (\textit{Pinus sylvestris}) is the most widely distributed pine species in the world. It's range spans Eurasia from the arctic circle in Scandinavia to the dry northern mediterranean in Spain and Turkey and to the eastern edge of Siberia \citep{Carlisle1968}. 

Scotland represents the western limit of its distribution, where it is the dominant canopy tree species of the Caledonian pine forest. Scots pine grows well under low grazing, shade and competition \todo{Matyas2004}. 

Scots pine is wind pollinated, with monoecious flowering beginning between the ages of 15 and 30. 

Previous studies have identified genetically distinct populations of Scots pine in Scotland, with populations becoming more genetically isolated over distance as \todo{you} move further west, against the prevailing wind direction \citep{}. 

\subsection*{Study sites}

Sites were situated within the range of the historical range extent of the Caledonian pine forest. Sites were chosen by accessibility and to ensure geographical isolation. Previous studies have shown cryptic genetic variation between these sites \citep{Donnelly2018} which supports the assertion that despite strong cross-pollination effects between the populations, some degree of genetic isolation occurs. Variation in isolatedness between sites follows a predictable longitudinal gradient, with sites on the western extreme of the Caledonian pine range being more isolated due to the prevailing West->East wind direction.

\subsection*{Experimental design}

Seedlings were grown for \todo{YEARS} in glass houses in \todo{CONDITIONS} before being planted in the common garden.

The common garden was located at \todo{COORDS} in a \todo{SIZE} patch of recently clear-felled sitka spruce plantation, surrounded by existing adult sitka spruce (\textit{Picea sitchensis}) plantation on all sides. All sitka spruce surrounding the common garden was planted at the same time \todo{WHEN}. Planting was divided into four blocks of equal size running perpendicular to the average slope of the site. Each block contained 167 saplings, with saplings placed in a grid pattern within each block, with a gap between each sapling of \todo{METRES} and between saplings and the plot edge \todo{FIGURE REF}. Saplings were randomly assigned to grid points.

Pine weevil infestation occurred naturally across the site, with adult weevils likely travelling from the adult sitka spruce plantation adjacent to the site. 

\subsection*{Statistical analysis}

To isolate the effects of sapling genetic origin resistance to damage by pine weevils, we conducted generalised least squares linear models using the `\textit{nlme}' package in R \citep{Pinheiro2018}. Zero-inflated count data 


All analyses were conducted in R version 3.4.2 \citep{RCoreTeam2017}.


\section*{Results}

\section*{Discussion}

\section*{Conclusion}

TEXT

%-----------------------------------------------
\end{document}
%-----------------------------------------------


% Think about framing the question in terms of identifying conservation priorities for genetically distinct populations which don't have much resistance to the weevils
