

% Report Template
% John Godlee (johngodlee@gmail.com)
% Compile with XeLaTeX

\documentclass[a4paper, 11pt]{article}
%
\usepackage{amsmath}   % Better maths and more symbols
%
\usepackage{geometry}  % Set margins
\geometry{left=1.2cm,
	  right=1.2cm,
	  top=2.2cm,
  	  bottom=2cm}
\parskip 0.5cm
%
\usepackage{pdflscape} % Allow landscape pages nested in pdf
%
\usepackage{graphics}  % Insert images easily
\usepackage{graphicx}
\graphicspath{ {img/} }
\usepackage{float}  % Fancy graphics placement [H] [H!] arguments
%
\usepackage{caption}
%
\usepackage{multirow}  % Table cells spanning multiple rows
%
\usepackage{enumerate}
%
\usepackage{natbib}    % Bibliography management - Use author/date citations
\bibliographystyle{agsmnourl}  % Use my custom agsm bibliography template which never includes URLs in articles
\usepackage{url}
\usepackage{cite}
%
\usepackage{lineno}
\linenumbers
%
\usepackage{eurosym}
\usepackage{textcomp}
%
\usepackage{color} 
\newcommand{\todo}[1]{\textcolor{red}{#1}}   % \todo{NOTE TO SELF WRITTEN IN RED}
%
\title{Geographically and genetically distinct populations of scots pine (\textit{Pinus sylvestris}) do not differ in resistance to damage the large pine weevil (\textit{Hylobius abietis}), a common garden translocation study}
\author{John L. Godlee}


%-----------------------------------------------
\begin{document}
%-----------------------------------------------

\maketitle{}

\begin{abstract}

Damage to coniferous plantation crops from the large pine weevil \textit{Hylobius abietis} causes economic losses of \euro{}140 million in Europe \textit{per annum}. Current mitigation strategies are labour intensive and only partially effective. Natural resistance of host plants to insect pests has been used in many crop species to reduce damage as part of an integrated pest management strategy. Here, we conducted a common garden experiment in a previously clearfelled forestry plantation where \textit{H. abietis} are known to occur. 672 seedlings, collected from 21 naturally occurring populations of \textit{Pinus sylvestris} across Scotland were planted in a random grid. 26\% of saplings were damaged by \textit{H. abietis} but there was no sapling mortality in the year of the study. We found significant variation in pine weevil lesion abundance between \textit{P. sylvestris} populations, suggesting that these individuals could be selected for selective breeding to produce trees that are less susceptible to \textit{H. abietis} attack at the sapling stage.

\end{abstract}

\section*{Introduction}

The large pine weevil (\textit{Hylobius abietis} L. Coleoptera: Curculionidae) is a common pest of newly planted conifer plantations in Europe, generally causing damage to saplings up to five years old \citep{Dillon2008}. Adult weevils emerge from tree stumps and feed on the bark and buds of coniferious saplings. Circular lesions on the bark and buds of saplings as a result of feeding may cause a reduction in growth rate, stem deformation and an increased susceptibility to infection by airborne diseases of trees \citep{Leather1999}. Heavy damage may lead to stem girdling, resulting in a malformed stem, limiting economic use as a timber tree when fully grown \citep{}. While pine weevils may inhabit adult coniferous trees in both natural and planted coniferous forests, recently clearfelled and restocked coniferous forest plantation sites provide enriched habitat for breeding pine weevils. Adults lay eggs within the stumps of clearfelled trees which are rarely removed after clearfelling. with newly emerged juvenile weevils feeding on young saplings planted for restocking, causing damage and mortality. A single adult weevil can damage several plants over the course of a season, with \textasciitilde{}50\% sapling mortality observed across infected plantation sites in the UK and Ireland \citep{Heritage2001}. On commercial conifer plantations, \textit{H. abietis} causes annual economic losses of \euro{}140 million \textit{per annum} in Europe, of which \euro{}2.75 million occurs in the UK \citep{Evans2015}. The potential for climate change to enhance the damage caused by \textit{H. abietis}, by reducing life cycle length \citep{} and encouraging migration into previously weevil free areas \citep{}, especially in more northerly areas such as northern Scotland, has prompted discussion of the effectiveness of current management practices and possible alternatives \citep{}.

Management of \textit{H. abietis} currently relies on a variety of chemical, biological and physical measures, with integrated pest management schemes tending to yield greater success \citep{}. Physical deterrents include piling debris produced by the clearfelling process over exposed stumps to discourage egg laying, or stump removal to limit the availability og substrate for egg laying. The application of entomopathogenic nematodes after clearfelling has been shown to reduce the number of adult weevils in clearfelled site \citep{Dillon2006}. The most common method of control is the addition of chemicals at the time of restocking, with \textit{H. abietis} being the only insect pest against which routine chemical controls are applied in the UK and Ireland \citep{Willoughby2004}.  

Scots pine (\textit{Pinus sylvestris} L.) is one of the UK's three endemic coniferous tree species. It constitutes \textasciitilde{}\todo{percentage} of the UK's commercial plantation forestry by biomass, while natural populations are restricted to enclaves in the north of Scotland. \todo{Caledonian remanents.} Pine weevils are known to attack scots pine plantations \citep{}. As \textit{P. sylvestris} is one of the UK's three native coniferous tree species \citep{}, and with an increasing interest in planting native tree species in an attempt to preserve native biodiversity and landscape heritage \citep{}, an increasing percentage of plantation forestry in the UK is planting of \textit{P. sylvestris}.

Rather than the application of costly pesticides or biological control agents, selective breeding of weevil resistant \textit{P. sylvestris} varieties, or identifying existing genetically distinct populations of \textit{P. sylvestris} which are resistant to weevil damage for commercial plantation forestry may provide a low cost method of reducing economic losses from weevil damage.

Breeding natural resistance to \textit{H. abietis} is being heavily explored with other tree species such as the Norway Spruce (\textit{}), but \textit{P. sylvestris} has not recieved the same attention. \todo{other crops where natural resistance has worked}

Planted \textit{P. sylvestris} saplings are more susceptible to \textit{H. abietis} damage, possibly due to water stress as a result of damage to root systems during planting \citep{Selander1990}.

We conducted a common garden experiment to assess the resistance of \textit{P. sylvestris} saplings grown from seed collected in populations across the natural longitudinal range of Caledonian Pine to damage from the large pine weevil \textit{H. abietis}. We hypothesised that 

\section*{Materials \& Methods}

\subsection*{Study species}

Scots pine (\textit{Pinus sylvestris}) is the most widely distributed pine species in the world. It's range spans Eurasia from the arctic circle in Scandinavia to the dry northern mediterranean in Spain and Turkey and to the eastern edge of Siberia \citep{Carlisle1968}. 

Scotland represents the western limit of its distribution, where it is the dominant canopy tree species of the Caledonian pine forest. Scots pine grows well under low grazing, shade and competition \todo{Matyas2004}. 

Scots pine is wind pollinated, with monoecious flowering beginning between the ages of 15 and 30. 

Previous studies have identified genetically distinct populations of Scots pine in Scotland, with populations becoming more genetically isolated over distance as \todo{you} move further west, against the prevailing wind direction \citep{}. 

\subsection*{Study sites}

Sites were situated within the range of the historical range extent of the Caledonian pine forest. Sites were chosen by accessibility and to ensure geographical isolation. Previous studies have shown cryptic genetic variation between these sites \citep{Donnelly2018} which supports the assertion that despite strong cross-pollination effects between the populations, some degree of genetic isolation occurs. Variation in isolatedness between sites follows a predictable longitudinal gradient, with sites on the western extreme of the Caledonian pine range being more isolated due to the prevailing West->East wind direction.

\subsection*{Experimental design}

Seedlings were grown for \todo{YEARS} in glass houses in \todo{CONDITIONS} before being planted in the common garden.

The common garden was located in Southern Scotland (N 55.86\textdegree{}, E −3.21\textdegree{}) in a \todo{SIZE} patch of recently clear-felled sitka spruce plantation, surrounded by existing adult sitka spruce (\textit{Picea sitchensis}) plantation on all sides. All sitka spruce surrounding the common garden was planted at the same time \todo{WHEN}. Planting was divided into four blocks of equal size running perpendicular to the average slope of the site. Each block contained 167 saplings, with saplings placed in a grid pattern within each block, with a gap between each sapling of \todo{METRES} and between saplings and the plot edge \todo{FIGURE REF}. Saplings were randomly assigned to grid points within blocks. This resulted in a total grid size of 84 x 8 saplings.

Pine weevil infestation occurred naturally across the site, with adult weevils likely travelling from the adult sitka spruce plantation adjacent to the site. 


\subsection*{Data collection}

Pine weevil lesions were counted on each sapling stem between \todo{DATES}. This is roughly between the two seasonal peaks of weevil feeding that are commonly observed in the UK, which occur in the spring and late summer, coinciding with \todo{MORE} \citep{}. Isolated lesions tended to be roughly circular with a diameter of \textasciitilde{}3 mm. Where a larger continuous legion, as when a stem was girdled, the larger lesion was photographed with a scale and the area estimated by tracing the lesion with Imagej version 1.50g7 \citep{Schneider2012}. Weevil damage is therefore expressed as the mm\textsuperscript{2} area of stem lesions per sapling. 

\subsection*{Statistical analysis}

To assess the effect of sapling genetic origin to damage by pine weevils, we conducted a two-step linear model procedure. First, a binomial logistic regression assessed the probability of a sapling being damaged by weevils across each site. Then a generalised least squares model using the \{\textit{nlme}\} package in R \citep{Pinheiro2018}, using only saplings where damage had occurred, assessed whether there was an effect of sapling genetic origin on the severity of damage. The response variable of these models was the number of pine weevil lesions visible on the sapling. Due to many saplings receiving no weevil damage, zero-inflated data was transformed by adding 1 to all measurements and log transforming. All statistical analyses were performed in R version 3.4.2 \citep{RCoreTeam2017}.

Spatial autocorrelation in number of lesions due to certain susceptible saplings acting as beacons to attract weevils was accounted for in the generalised least squares model by including a spatial autocorrelation structure in the generalised least squares models. Multiple spatial autocorrelation structures were tested and models were compared in their goodness-of-fit using Akaike Information Criterion (AIC) values and Log-likelihood estimates \citep{}. 

After model selection, the best generalised least squares model was re-fitted using Restricted Maximum Likelihood (REML) for model interpretation.

\section*{Results}

\section*{Discussion}

\section*{Conclusion}


%-----------------------------------------------
\end{document}
%-----------------------------------------------


% Think about framing the question in terms of identifying conservation priorities for genetically distinct populations which don't have much resistance to the weevils
% Alright idea, but natural population of scots pine really don't have that much damage from pine weevils, mainly because there aren't an abundance of stumps and young trees close together in natural populations.
