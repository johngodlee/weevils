\documentclass[a4paper,12pt,]{article}
\usepackage{titling}
\usepackage{fullpage}
\usepackage{changepage}
\usepackage{amsmath}
\usepackage{cite}
\usepackage[margin=1.5cm]{geometry}
\usepackage{setspace}
\usepackage[compact]{titlesec}
\usepackage{mdwlist}
\usepackage{hyphenat}
\usepackage[official]{eurosym}
\usepackage{textcomp}
\usepackage{graphics}
\usepackage{graphicx}
\usepackage{caption}
\usepackage{natbib}

\bibliographystyle{agsm}
\begin{document}


\titlespacing{\section}{0pt}{*3}{*-0.1}
\titlespacing{\subsection}{0pt}{*-0.5}{*0}
\titlespacing{\subsubsection}{0pt}{*0}{*0}

\raggedright{{\large{Do geographically distinct populations of Scot's Pine (\textit{Pinus sylvestris}) differ in resistance to damage by the Pine Weevil (\textit{Hylobius abietis})? A common garden translocation study.}}}

\section*{\large{Background}}
The pine weevil (\textit{Hylobius abietis}) is a common pest of newly replanted conifer plantations in Europe, generally causing damage to saplings up to  5 years old \citep{Dillon2008}. Adult weevils emerge from tree stumps and feed on the bark and buds of coniferous saplings. Side-effects of this damage include a reduction in growth rate, stem deformation, and increased susceptibility to infection \citep{Leather1999}. Without effective treatment by increasingly expensive and restricted insecticides \citep{Osteen2002}, \textit{H. abietis} could cause economic losses of up to \euro{}140,000,000 per year in the European forestry industry \citep{Langstrom2004}. As such, the control of \textit{H. abietis} by alternative methods is ongoing \citep{Dillon2008, Ennis2010, Luoranen2012, Nordlander2009}. 
\\[0.3cm]
Rather than modifying the environment around saplings (\textit{i.e.} by application of insecticides or deterrents), this study will investigate the passive effects of ecotype on the susceptibility of Scot's Pine (\textit{P. sylvestnis}) saplings to damage by \textit{H. abietis}. 
\\[0.3cm]
This study forms part of a larger investigation into the local adaptation of naturally seeded \textit{P. sylvestnis} populations in Scotland. Seeds were collected from 21 populations of  \textit{P. sylvestnis} throughout Scotland and grown in greenhouses until they were considered robust enough to be planted. 3 sites in Scotland were chosen for planting, selected for their vastly differing environmental characteristics. The site at Yair Forest near Selkirk, which has a high abundance of pine weevils, is considered here.


\section*{\large{Sampling strategy}}
Saplings from 21 populations were planted in a grid throughout a plot in the Yair Forest, with 3 m between saplings. Sapling distribution is fully randomised within 4 blocks to account for environmental variation over the site. Each population is represented by eight families (21$\times$8=168 families) from which the seeds were collected, each family has one seedling per block, a total of 672 saplings. All the saplings within the site will be measured for the extent of  weevil damage during May/June, when weevil activity is at its greatest \citep{Wainhouse2007}.
\\[0.3cm]
The main stem (where most damage occurs) of each sapling will be divided into 10 equal length zones, measuring from the ground to the main stem apex. Zones reduce subjectivity by demanding equal attention to all parts of the main stem. In each zone a five point scale will assess overall weevil damage in terms of percentage area of bark removed, the sum of these scores will be expressed as a percentage of the total damage possible \textit{i.e.} 40 points (Table 1) \citep{Zas2005}. This subjective measurement should be conducted by the same researcher for all trees in order to standardise the measurements . Stem damage will be categorised into this year's damage and last year's damage in order to assess any relationships over time, \textit{i.e.} saplings which were heavily damaged last year may be more susceptible or more resistant to subsequent attacks. A possible source of error exists whereby damage from the previous year is under-represented due to the scar being attacked again this year.
\\[0.3cm]
%The depth of the wound also influences the level of long term damage to the sapling. Superficial wounds do not generally cause long term damage as xylem vessels are not severed, deeper wounds however, which may have the same area as a superficial wound, can cause permanent deformation of the stem and can lead to branch death. Therefore, the largest discrete wound in each zone will be evaluated as to whether it is superficial or deep (Wainhouse \textit{et al.}, 2007).

\begin{table}[h]
\caption{Scoring system for area of bark removed per zone}
\centering
\begin{tabular}{|r|r|}
\hline
$Score$&$Interpretation$\\
\hline
0&No damage\\
1& 0-25\% bark removed\\
2& 25-50\% bark removed\\
3& 50-75\% bark removed\\
4& 75-100\% bark removed\\
\hline
\end{tabular}
\end{table}

The death of the main stem above a deep wound is common due to a severing of vascular tissue and the resulting lack of water. Main stem death as a result of weevil damage (\textit{i.e.} directly above a circumferential wound) will be expressed as a percentage of the total height of the sapling \citep{Zas2005}.
\\[0.3cm]
The presence of resin at the wound surface is thought to confer resistance by preventing further wounding in the same area, preventing infection by pathogens and promoting the regrowth of bark \citep{Wainhouse2007}. Non-resistant individuals produce no resin at the surface, while resistant trees may only produce some resin depending on the time interval between weevil attack and measurement. Therefore the presence of resin will be noted true or false for each sapling as a whole.
\\[0.3cm]
Bud damage will be classified into superficial bores and larger wounds. Superficial wounds are only as wide as the mouthparts of a weevil and represent investigatory bites, whereas larger wounds represent continued feeding and cause major tissue loss, often leading to bud death, The number of wounds will be measured on all terminal buds. The number of terminal buds will be measured. Example wounds should be identified and used as reference points when classifying other wounds.
\\
\section*{\large{Data analysis}}
Anderson-Darling normality tests will assess the normality of the bark-area-removed data for all saplings and the main-stem-death data. Transformations will be made accordingly to gain normality of residuals. 
\\[0.3cm]
ANOVA analyses will assess the effect of population on the percentage area of bark removed, the effect of population on main stem death, and the effect of population on bud damage under the null hypothesis that there is no effect of population. Spatial blocks will be included as a second categorical variable to investigate possible effects of location on weevil damage. Buds and stems which have been damaged by anything other than pine weevil feeding (\textit{i.e.} herbivorous mammals, wind etc.) will not be included in the analysis. Assuming significant results (P\textless0.05), pairwise Tukey's tests may assess which populations differ significantly (P\textless0.05). 
\\[0.3cm]
A Chi-squared test ($\chi$\textsuperscript{2}) will assess whether there is an effect of population on the presence of resin in each sapling. Individuals with no wounding will not be included in the analysis.
\\[0.3cm]
A Pearson's Moment Correlation Coefficient will assess the relationship between the area of bark removed this year and the area removed last year.
\footnotesize{\bibliography{WeevilSampling}}
\end{document}